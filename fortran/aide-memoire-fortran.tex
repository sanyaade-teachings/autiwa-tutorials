\documentclass[a4paper,twoside]{article}
\usepackage{autiwa}
\usepackage{listings}
\lstset{language=Fortran,
basicstyle=\ttfamily\small,
columns=flexible,
escapechar=+}
\lstset{keywordstyle=\color{blue}\bfseries}

\title{Aide mémoire Fortran 90}
\author{Autiwa}

\newcommand{\raccourci}[1]{{\bfseries #1}}

\makeindex
\begin{document}

\tableofcontents

\clearpage

\section{Préambule}
Ceci est un tutoriel fortran 90, il a pour but de donner des astuces de programmations, des bonnes pratiques, présenter ce qui se faisait en fortran 77 et qu'il ne faut plus faire. 

Dans la suite on considèrera le format libre, c'est à dire que les lignes peuvent avoir jusqu'à 132 caractères.

\section{Transition fortran 77/fortran 90}
\subsection{Instructions obsolètes ou dépréciées}

\begin{center}
\begin{tabular}{ll}
Obsolètes & Déprécié\\
IF arithmétique & format fixe\\
GO TO assigné & COMMON\\
RETURN multiple & DATA au milieu des inst.\\
FORMAT assigné & BLOCK DATA\\
DO sur une même instruc. & EQUIVALENCE\\
Index réel de boucle DO & GO TO calculé\\
branchement sur END IF & INCLUDE\\
PAUSE & ENTRY\\
descripteur H & DOUBLE PRECISION\\
 & Instructions Fonction\\
 & SEQUENCE\\
 & DO WHILE
\end{tabular}
\end{center}

\section{Les bases}
\subsection{Éléments de syntaxe}

Pour continuer une ligne, en cas de ligne trop longue : 
\begin{lstlisting}[language=Fortran]
print *, 'Montant HT :', montant_ht, & '	TVA:',tva	,&
'Montant TTC :', montant_ttc
\end{lstlisting}

\bigskip

Pour continuer une chaîne de caractère par contre, il faut impérativement utiliser deux caractères \og \& \fg : 
\begin{lstlisting}[language=Fortran]
print *, 'Entrez un nombre entier & 
	&compris entre 100 & 199'
\end{lstlisting}

\bigskip

Les commentaires commencent par le symbole \og ! \fg : 
\begin{lstlisting}[language=Fortran]
if (n < 100 .or. n > 199) ! Test cas d'erreur
! On lit l'exposant
read *,x 
! On lit la base
read *,y 
if (y <=0) then	! Test cas d'erreur 
  print *,' La base doit être un nombre >0'
else 
  z = y**x	! On calcule la puissance
end if
\end{lstlisting}

\bigskip

\end{document}