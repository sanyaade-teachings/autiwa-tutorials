\documentclass[a4paper,twoside]{article}
\usepackage[french]{autiwa}
\title{Embryos assembly in a radiative disk}
\author{C. Cossou, S. Raymond \and A. Pierens}

\begin{document}
\tableofcontents
\newpage

\section{Introduction}
La formation des planètes autour de leur étoile est un sujet complexe et étudié depuis de nombreuses années. Sa complexité est telle que de nombreuses approximations sont nécessaires, parfois justifiées, parfois non. La récente complexification des interactions entre le disque et les embryons par la levée d'une de ces approximations (disque isotherme) a amené la dynamique des planètes dans le disque à un niveau extrêmement élevé, tous les effets étant intimement liés. 

Dans les parties qui vont suivre, je vais essayer de décrire le plus clairement possible les enjeux et les limites de chacun des grands sous domaines liés à la formation planétaire, en particulier concernant les propriétés du disque protoplanétaire dans lequel elles se forment

\section{Les disques de gaz}
\subsection{Formation et évolution d'un disque protoplanétaire}
%TODO parler de l'évolution du disque, sa dissipation etc...

\subsection{Approximations et paramètres de modélisation}
%TODO parler du profil de densité, 
%TODO parler de la température du disque (et les phénomènes principaux qui ont un effet sur la température, chauffage visqueux, irradiation de l'étoile, irradiation externe. Parler dans cette partie de l'opacité, des transitions et à quoi c'est dû, des modèles had oc pour l'opacité et des incertitudes qui en découlent

%TODO parler des bords du disque et de tous les problèmes que ça pose

\section{La dynamique planétaire}
\subsection{Les résonances de Moyen Mouvement (MMR)}
%TODO parler des résonances de moyen mouvement


\section{Les interactions entre le disque et les planètes}
\subsection{La migration}
%TODO Parler du couple exercé par le disque sur les planètes (lindblad et corotation, parler des différents couples de corotation, que je ne connais pas pour le moment

\subsection{L'accrétion du gaz}
%TODO parler de l'accrétion, et du fait que ça va créer des planètes géantes notamment

\subsection{L'amortissement de l'excentricité}%circularisation
%TODO parler des autres phénomènes importants dans le disque, comme l'amortissement de l'excentricité

\subsection{L'amortissement de l'inclinaison}%coplanarisation
%TODO parler de l'amortissement de l'inclinaison, 








%TODO du couple de corotation quand l'excentricité augmente, mais je sais pas encore où le placer, pas dans la partie intro je suppose

Liste en vrac de choses à dire : 
Quel code?

Ce que j'ai rajouté? 

Ce que j'ai modifié à paardekooper pour le rendre mieux (profil de température par exemple)

zone de convergence : les embryons se rassemblent, mais sont bloqués par les résonances

présentation de mes diagrammes de couple (que je présenterai plus rapidement ensuite, mais les présenter de manière générale et détaillée.)

effet des résonances sur la migration (les embryons se stabilisent, et on peut considérer le système planétaire comme un ensemble dont le couple de migration est nul

effet de bitsch et kley sur la migration : ça décale la zone de convergence (cf papier 1)

effet du type de résonance sur la dynamique planétaire (une 3:2 par rapport à une 7:6, ça change quoi normalement, et ça change quoi dans un disque quand il y a la migration en plus, détailler les deux cas



effet des paramètres du disque : 
* viscosité
* profil de température
* profil de densité de surface
* masse du disque (à profil constant). 
* table d'opacité

Il faudrait que je trouve toutes les incertitudes qu'il y a dans mon modèle : 
* la table d'opacité
* ma viscosité constante
* les bords du disque
* la densité volumique qui n'est dans mon code que la densité surfacique divisée par l'échelle de hauteur
* pas d'effets indirects des ondes de densité d'une planète sur les autres planètes (effet visible en hydro, mais qui n'apparait pas ici en N-corps)

\chapter{Simulations}
\section{Mécanisme d'accrétion à une zone de convergence}

\section{Différents types de zone de convergence}

\section{Les résonnances comme moteur de la non-circularisation des orbites}

\section{Rétroaction de l'excentricité sur la migration planétaire}

\section{Influence des paramètres du disque}

\end{document}
