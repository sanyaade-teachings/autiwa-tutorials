\section{Étude de sensibilité}
%TODO 
\subsection{Le choix de la table d'opacité et son implémentation}
%TODO le choix de la table, mais aussi le fait qu'on a besoin d'une densité volumique, ou qu'on a besoin de la masse moléculaire moyenne. 

\subsection{Modélisation de la viscosité}
%TODO 



\section{Approximations}
%TODO 
\subsection{Profil de densité du gaz en 2D}
%TODO Nous sommes dans un profil (1+1D) c'est à dire qu'il n'y a pas d'intéraction radiale/verticale dans notre disque. Les valeurs sont moyennées en particulier pour le calcul de la température, on ne résoud pas l'équilibre hydrostatique. 
%TODO 
\subsection{La modélisation des bords du disque}
%TODO 
\subsection{Pas d'effet indirect des ondes de densité sur les autres planètes}
%TODO 
\subsection{Auto-gravité}
%TODO 

\section{Idées}
%TODO
\subsection{Snow line comme source de particules}
%TODO commenter l'article ci-dessous
%Title: Ice condensation as a planet formation mechanism
%Authors: Katrin Ros and Anders Johansen
%Categories: astro-ph.EP
%Comments: 15 pages, 11 figures, submitted to A&A, version includes revisions in
% response to referee report