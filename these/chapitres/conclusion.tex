J'ai cherché tout au long de ma thèse à développer un code numérique simple et modulaire avec une idée en tête, pouvoir tester les interactions et/ou différences entre différents modèles. L'idée était de profiter des libertés offertes par un code N-corps plus rapide pour tester des domaines de l'espace des paramètres qui ne sont pas accessibles aux codes hydrodynamiques. 

\bigskip

Dans le chapitre 3, j'ai étudié la migration dans les disques à l'aide de cartes de migration qui permettent de comprendre le comportement de la migration dans un disque donnée de manière assez intuitive. Dans un premier temps j'ai cherché à comprendre la forme de ces cartes de migration. Puis, j'ai cherché à comprendre l'influence des paramètres sur disque sur le migration, toujours à l'aide des cartes de migration. 

En particulier, la viscosité et la densité de surface ont une influence très importante sur la carte de migration. Ces grandeurs, bien que dépendant intrinsèquement du disque considéré, vont aussi varier au cours de la dissipation du disque. Comprendre l'évolution de la migration en fonction de ces paramètres nous permet donc de mieux comprendre comment va se comporter une planète au cours de la vie du disque. Au cours de la dissipation du disque, la viscosité va, au même titre que la densité de surface, diminuer au cours du temps \citep[Fig. 16]{guilloteau2011dual}. La décroissance de ces deux paramètres, prise séparément, va dans le même sens, c'est à dire le déplacement des zones de convergence vers les parties internes du disque. De plus, au fur et à mesure de la dissipation du disque, les zones de convergences disparaissent.

La découverte par hasard de l'interaction entre résonance et amortissement du couple de corotation en est un exemple. L'effet de l'excentricité sur le couple de corotation a été montré pour la première fois par des simulations hydrodynamiques 3D\citep{bitsch2010orbital}. Modélisé et simplifié pour pouvoir l'appliquer dans des simulations N-corps, un effet supplémentaire a été étudié dans le cadre de systèmes multiplanétaires : la stabilisation d'un système autour d'une zone d'équilibre qui ne correspond à aucune zone de couple nul dans le disque. 

\bigskip

La principale chose que montre le chapitre 3 est la diversité des disques et des migrations qu'ils abritent. Les populations synthétiques de planètes ne peuvent reproduire la population de planètes extrasolaires avec un seul type de disque. Les propriétés de l'étoile, la masse du disque, la quantité de poussière, la dissipation du disque et l'évolution subséquente de la densité de surface, l'opacité et la température. 

L'autre chose que cela montre, c'est que nous manquons cruellement de contraintes observationnelles. En particulier sur la densité de surface. Le profil de la nébuleuse solaire minimale en $R^{-\frac{3}{2}}$ est largement utilisé, mais ne correspond pas aux observations qui trouvent un profil moyen en $R^{-1}$. De plus, il est peu probable qu'une seule loi de puissance décrive correctement la totalité d'un disque. Le problème c'est que les modèles actuels ne peuvent pas modéliser de manière cohérente la viscosité et les profils de densité de surface et de température sans faire d'approximations. Suivant les modèles, c'est l'un des trois paramètres qui devient un paramètre libre. 

Les observations, quant à elles, déterminent ces mêmes grandeurs par d'autres modèles. La masse du disque étant déterminée au travers des densités de colonnes de différentes molécules, des modèles d'opacité sont alors utilisés afin de revenir au profil de température, puis de densité de surface. La masse du disque est alors déterminée à partir des parties externes du disque, uniques parties accessibles aux observations. La masse totale du disque est extrapolée en supposant une loi de puissance pour la densité de surface. 

ALMA devrait amener des observations de très haute résolution de disques protoplanétaires, et un gain de précision dans toutes contraintes liées aux disques. Ces contraintes devraient nous permettre de limiter l'espace des paramètres et de trouver quelques disques dits \og classiques\fg sur lesquels la formation planétaire pourra se concentrer. 

Il n'est pas impossible que dans le futur, certaines sous-populations particulières d'exoplanètes puissent être expliquées par un type particulier de disque, éloigné des disques classiques. En effet, si le nombre de planètes est faible par rapport au nombre total, rien n'exclut ce genre de scénario, au même titre que la formation des planètes du système solaire n'a aucune raison de ne pouvoir invoquer des évènements particuliers et improbables. 

