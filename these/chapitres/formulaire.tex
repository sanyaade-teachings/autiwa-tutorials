\chapter{Formulaire}
Ici sont répertoriées bon nombre de formules que j'ai utilisé et qui relient des grandeurs physique entre elles. Dans la mesure du possible, une source est donnée où la formule est mentionnée. Ceci a pour but de centraliser ces formules, liées à la physique des disques, et que j'ai parfois eu du mal à retrouver parmis la quantité d'articles ou de livres traitant du sujet. 

\section{Variables usuelles}\label{sec:variables}
Dans la thèse, j'utilise couramment les mêmes notations pour une propriété physique donnée. Ici je fais un inventaire des notations, afin qu'on puisse s'y référer, et pour gagner en clarté dans le texte en m'évitant de redéfinir à chaque fois les mêmes unités :

\begin{table}[htb]
\centering
\begin{tabular}{|>{$}c<{$}|p{7cm}|}
\hline
\nu & Viscosité du disque\\\hline
b/h & Longueur de lissage du potentiel gravitationnel de la planète en unité de son rayon de Hill\\\hline
c_s & vitesse du son\\\hline
\alpha & paramètre adimensionné pour la prescription $\alpha$ du disque, permettant de définir une viscosité fonction de la vitesse du son\\\hline
H & Échelle de hauteur du disque\\\hline
h=H/R & rapport d'aspect du disque\\\hline
\Omega & vitesse angulaire d'une particule fluide ou d'une planète dans le disque\\\hline
k_B & constante de Boltzmann \\\hline
T & Température\\\hline
\mu & Masse moléculaire moyenne du gaz constituant principal du disque\\\hline
\Sigma & Densité de surface du disque de gaz\\\hline
\rho & Densité volumique du disque de gaz\\\hline
q & rapport adimensionné entre la masse de la planète et la masse de son étoile\\\hline

\end{tabular}
\caption{Liste de la plupart des variables utilisées tout au long de la thèse. Les paramètres avec un $p$ en indice indiquent simplement que c'est la valeur du paramètre à la position orbitale de la planète.}
\end{table}

\section{Propriétés du disque}

La prescription alpha pour la viscosité d'un disque est définie par :
\begin{align}
\nu &= \alpha c_s H
\end{align}

\begin{align}
c_s &= \sqrt{\frac{k_B T}{\mu m_H}}
\end{align}

\begin{align}
H &= \inv{\Omega}\sqrt{\frac{k_B T}{\mu m_H}}\\
&= \frac{c_s}{\Omega}
\end{align}
où $m_H$ est la masse d'un atome d'hydrogène.

On considère que la densité de surface est égale à la densité volumique, intégrée sur la taille verticale $2H$ du disque. 
\begin{align}
\Sigma &= 2\rho H
\end{align}

\section{Propriétés des orbites képleriennes}
Soit une planète de demi-grand axe $a$, d'excentricité $e$, de masse $m_p$, de période orbitale $T$ orbitant autour d'une étoile de masse $m_\star$. 

Il y a une relation entre sa période orbitale $T$ et son demi-grand axe :
\begin{align}
\frac{T^2}{a^3} &= \frac{4\pi}{G(m_\star + m_p)}
\end{align}
où $G$ est la constante de gravitation universelle.

On défini le périastre $q$ et l'apoastre $Q$ comme étant les distances minimales et maximales entre l'étoile et la planète : 
\begin{subequations}
\begin{align}
q &= a (1 - e)\\
Q &= a (1 + e)
\end{align}
\end{subequations}

La vitesse angulaire moyenne $\Omega$ (ou instantanée en supposant que $e\ll 1$) est définie par : 
\begin{align}
\Omega &= \sqrt{\frac{G(m_\star + m_p)}{a^3}}
\end{align}

La vitesse linéaire moyenne $v$ est définie par : 
\begin{align}
v &= \sqrt{\frac{G(m_\star + m_p)}{a}}
\end{align}

L'énergie $E$ et la norme du moment cinétique $J$ d'une orbite képlerienne de demi-grand axe $a$ et d'excentricité $e$ sont donnés par :
\begin{align}
E &= \inv{2}v^2 - \frac{G(m_\star + m_p)}{r} = \frac{G(m_\star + m_p)}{2a}\\
\norm{\vect{J}} &= G(m_\star + m_p) a (1-e^2)
\end{align}