\documentclass[a4paper,twoside]{article}
\usepackage{autiwa}
\usepackage{listings}



\title{Aide mémoire C++}
\author{Autiwa}

\newcommand{\raccourci}[1]{{\bfseries #1}}


\makeindex
\begin{document}

\tableofcontents

\clearpage

\section{Pour commencer}
Il est possible d'utiliser un IDE pour coder en C++. L'IDE multiplateforme que j'utilise est \textbf{code::blocks} que l'on 
peut installer sous linux par :
\begin{verbatim}
sudo apt-get install codeblocks
\end{verbatim}

Pour compiler en ligne de commande, on peut utiliser \textbf{g++} sous linux (le gcc pour c++) :
\begin{verbatim}
g++ -o monprogramme monprogramme.cpp
\end{verbatim}

\subsection{Code de base}
On appelle des librairies avec :
\begin{verbatim}
#include <iostream>
\end{verbatim}

\bigskip

On déclare aussi l'espace de nom dans lequel on ira chercher les variables. Ceci est utile dans le cas où une fonction serait 
déclarée dans plusieurs bibliothèques. 
\begin{verbatim}
using namespace std;
\end{verbatim}

\bigskip

Tout programme possède (et doit posséder) une fonction \textbf{main}. Cette fonction sera exécutée automatiquement et c'est à 
partir de cette dernière que tout le programme est lancé. 
\begin{verbatim}
int main()
{

  return 0;
}

\end{verbatim}

\subsection{Affichage}
La librairie \textbf{iostream} contient les fonctions pour les flux d'entrées/sorties. Les fonctions décritent ici sont donc 
contenues dans \textbf{iostream}.

Pour afficher du texte à l'écran, il faut faire :
\begin{verbatim}
cout << "Hello world!" << endl;
\end{verbatim}
ou de manière équivalente :
\begin{verbatim}
cout << "Hello world!\n";
\end{verbatim}

On a \verb|"\n"|$\Leftrightarrow$\verb|endl|. 

\end{document}
